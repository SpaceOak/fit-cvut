\section{Údržba}
  \begin{description}
    \item[Údržba software] Systematický rozvoj produktu s aplikací drobných změn. Během údržby se také odstraňují chyby a problémy.
  \end{description}

  \begin{itemize}
    \item Systém byl dodán v rozsahu dle nabídky
    \item Systém byl akceptován a rutinně provozován
    \item Systém neobsahoval příliš mnoho chyb
  \end{itemize}

  \subsection{Typy údržby}
    Směrnice ISO/IEC 14764 definuje následující typy údržby:

    \begin{description}
      \item[Corrective] Za účelem opravy nalezených chyb a problémů.
      \item[Adaptive] Za účelem udržení použitelnosti SW v měnícím se prostředí.
      \item[Perfective] Za účelem zlepšení výkonnosti nebo udržovatelnosti.
      \item[Preventive] Za účelem detekce a opravy latentních chyb než se stanou skutečné.
    \end{description}

  \subsection{SW development life cycle}
    Používá se \emph{Miniwaterfall}. O systému má dodávající velkou znalost a změny jsou typicky malého rozsahu.
    Velmi efektivní způsob.

  \subsection{Měření}
    Velmi snadno se získávají přesná čísla. Měření slouží jako podklad pro servisní smlouvu na další léta.

  \subsection{Dokumentace}
    Udržuje se software, potencionálně dlouho po vytvoření. Je potřeba mít specifikaci k uhlídání rozsahu.
    Znát architekturu aby bylo možné ji dodržet. Jednotlivé změny je také potřeba zdokumentovat.

  \subsection{Vývojové prostředí}
    Využívá se prostředí podobné produkci, které se i podobně používá. Využívá se CI a smoked tests.

  \subsection{Architektura}
    Architektura musí být navržena tak, aby byla schopna absorbovat nové požadavky.

  \subsection{CM}
    Evidují se všechny požadavky zákazníka. Je striktně proces změnového řízení.

  \subsection{Testy}
    Komplexní systémy se většinou po každé změně netestují celé, buď se netestují vůbec nebo:
    \begin{itemize}
      \item existují regresní automatické testy,
      \item průběžně se kontroluje každá chyba,
      \item testy jsou naplánované a zorganizované,
      \item existuje záznam o testování,
      \item existují akceptační testy a jejich záznamy.
    \end{itemize}

    \subsection{Tým}
      Tým údržby je založen na lidech, kteří ho prvotně vyvíjeli. Z důvodu únavy se lidé obměňují až se vytvoří úplně nový tým.
      Nové členy je tedy potřeba řádně zaučit.

    \subsection{Ekonomika}
      Cílem je, aby údržba byla výdělečná. Je potřeba mít efektivní proces a velmi přesné odhady.

    \subsection{Odhady}
      Je potřeba udělat mnoho odhadů. Při odhadech je nutné být konzistentní, ideálně používat stále stejnou metodiku odhadů.
      Odchylky je potřeba umět zdůvodnit.

    \subsection{Technická podpora}
      \begin{description}
        \item[Technická podpora] Služba zajišťující pomoc uživatelům při řešení chyb.
      \end{description}

      Rozlišuje se na 3 úrovně: L1, L2 a L3.

      \begin{description}
        \item[L1] \emph{First-line-support}. Styčný bod mezi uživateli a techniky. Náplní práce je radit uživatelům s obsluhou systému a přijímání chybových hlášení.
        \item[L2] Administrátorská podpora. Technici řeší incidenty systému - zpravidla ještě před náhlášením díky monitoringu.
        \item[L3] Expertní podpora. Technici mají expertní znalost systému - často přímo vývojáři.
      \end{description}

      \subsubsection{Placená podpora}
        Z pohledu zákazníka:
        \begin{itemize}
          \item Zajištění opravy chyb i po skončení záruky
          \item Minimalizace nedostupnosti služeb
        \end{itemize}

        Z pohledu dodavatele:
        \begin{itemize}
          \item Zajištění příjmu
          \item Pokud firma vyvíjí kvalitní SW, lze zpravidla efektivně poskytovat
          \item Možnost snáze reagovat na změnové požadavky (technici mají detailní znalost systému)
        \end{itemize}

      \subsubsection{SLA - Service level agreement}
        V rámci provozu systému a jeho podpory jsou garantovány určité parametry.
        Za nesplnění nebo vybočení z domluvených mezí jsou udělovány sankce.

        Typické parametry jsou:
        \begin{itemize}
          \item dostupnost,
          \item doba reakce,
          \item doba za kterou musí být chyba vyřešena.
        \end{itemize}

      \subsubsection{Rozdělení}
        Podle času:
        \begin{itemize}
          \item 24/7 - nepřetržitá,
          \item 8x5 - pouze v pracovní době,
          \item 10x5 - rozšířená pracovní doba.
        \end{itemize}

        Podle intenzity:
        \begin{itemize}
          \item On-site - dostupná přímo u zákazníka. Typicky L1.
          \item On-call - podpora po telefonu (levnější varianta).
        \end{itemize}

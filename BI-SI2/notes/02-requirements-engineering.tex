\section{Requirements engineering}
  \begin{description}
    \item[Requirements engineering] Proces, který pokrývá veškeré aktivity spojené s návrhem,
    dokumentací a údržbou požadavů na systém.
  \end{description}

  \subsection{Schématický pohled - fáze}
    \begin{itemize}
      \item Elicitation (schůzky, jednání, připomínkování, pozorování uživatelů, ...)
      \item Analysis (přemýšlení, vymýšlení, debaty a poznámky)
      \item Specification (dekompozice, psaní a používání notace)
      \item Verification (čtení textů, schůzkym jednání, poromítání GUI, rozsah, ...)
    \end{itemize}

  \subsection{Význam procesu}
    \begin{itemize}
      \item Špatně definované požadavky způsobují neúspěch projektů
      \item Specifikace poždavků znamená:
        \begin{itemize}
          \item Zadání práce pro techniky
          \item Specifikace toho, co bude dodáno
          \item Základní část dokumentace
        \end{itemize}
      \item Rozsah projetku je brán jako parametr ceny
    \end{itemize}

  \subsection{Základní pojmy}
    \begin{description}
      \item[Rozsah (scope)] Množství práce (typicky včetně dodávky systému).
      \item[Nabídka] Obsahuje definici rozsahu.
      \item[Specifikace požadavků] Dokumenty obsahující požadavky na systém. Vytváří se až po nabídce.
    \end{description}

  \subsection{Typy požadavků}
    Je potřeba myslet na všechny typy požadavků.
    Dotazy je potřeba směřovat na relevantní skupiny zainteresovaných osob.
    \begin{itemize}
      \item Požadavky na vlastní funkce a rozhranní
      \item Požadavky na rozhraní (uživatelské, SW, HW, grafické, komunikační, ...)
      \item Nefunkční požadavky (výkon, bezpečnost, spolehlivost, dostupnost, škálovatelnost)
      \item Ostaní požadavky
        \begin{itemize}
          \item Legislativní
          \item Vícejazyčnost
          \item Technologie
          \item Platforma
        \end{itemize}
    \end{itemize}

    \subsubsection{Požadavky na požadavky - dle IEEE}
      \begin{itemize}
        \item Correct - Přesně popisuje chování systému.
        \item Unambiguous - Nemělo by možné si je různak vyložit.
        \item Complete - Obsahuje úplné požadavky na funkčnost systému a jeho vlastnosti.
        \item Consistent - Požadavky se navzájem nevylučují.
        \item Ranked - Třídění podle důležitosti. Každý požadavek je řazen podle důležitosti a kritičnosti.
        \item Verifiable - Musí být možné ověřit, že byl požadavek naplněn.
        \item Modifiable - Je možné požadavek upravit.
        \item Traceable - Sledování změn požadavků. Je nutné mít zaznamenanou každou změnu.
      \end{itemize}

\section{Architektura a design}
  \begin{description}
    \item[Software architecture] Popisuje programovací paradigmata,
    architektonické styly, principy a standardy. Definuje komponenty a vztahy mezi nimi.
    \item[Software design] Realizace funkčních požadavků. Jedná se o návrhové vzory, programovací idiomy
    a refaktoring. Společně s architekturou udává postup implementace.
    \item[Business process architecture] Zaměřuje se na obchodní strategie,
    řídící procesy a organizační struktury.
    \item[Information technology architecture] Zkoumá HW a SW infrastrukturu
    organizace nutnou pro chod systému. Umožňuje identifikovat HW a SW požadavky.
    \item[Information architecture] Zkoumá uložiště a databáze zpracovávaných
    dat, jakým způsobem jsou ukládána, používána a zpracovávána.
  \end{description}

  \subsection{Design}
    S designem jsou spjaté koncepty:
    \begin{itemize}
      \item dekompozice,
      \item abstrakce,
      \item zapouzdření,
      \item koheze,
      \item vazby.
    \end{itemize}

    Pojmy spojené s designem:
    \begin{itemize}
      \item abstraktní datový typ,
      \item typ,
      \item třída,
      \item objekt,
      \item instance,
      \item modul.
    \end{itemize}

  \subsection{Architektonické styly}
    Popisují způsobym jak navrhovat moduly a komunikaci mezi nimi. Mezi tyto styly patří:

    \begin{itemize}
      \item MVC - Model-View-Controller,
      \item MVP - Model-View-Presenter,
      \item Event driven architecture,
      \item Layered architecture,
      \item Repositories.
    \end{itemize}

  \subsection{Pohled na architekturu}
  \begin{description}
    \item[Logický] Poskytuje abstraktní pohled na problém. Obsahuje diagramy tříd, ale neobsahuje implementaci.
    \item[Modulární (Development)] Člení problém do modulů a subsytémů. Definuje jak postupovat při implementaci
    (jazyk, architekturu,..).
    \item[Koordinační (Process)] Zohledňuje spolupráci a synchronizaci procesů. Definuje výkon, dostupnost, toleranci výpadku a integritu.
    \item[Fyzický] Definuje škálovatelnost, výkon a dostupnost. Obsahuje popis mapování SW na HW. Snaha o nízký dopad na zdrojové kódy.
  \end{description}

  \subsection{Frameworky}
    \begin{itemize}
      \item Znovupoužitelney návrh pro SW systém
      \item Základ při vývoji jiných SW aplikací
      \item Diktuje architekturu systému
    \end{itemize}

  \subsection{Integrace}
    \begin{itemize}
      \item Spojeno s tématikou enterprise architektury
      \item Netechnologické (procesy, entity)
      \item Používání buzzwords (EAI, SOA, MOM)
    \end{itemize}

  \subsection{Shared database}
    \begin{itemize}
      \item Více aplikací sdílí společnou databázi
      \item Odpadají problémy se synchronizací
      \item Je ale potřeba vytvořit unifikované schéma pro všechny aplikace
      \item Úzké hrdlo výkonnosti
    \end{itemize}

  \subsection{Remote procedure call}
    \begin{itemize}
      \item Aplikace vlastní data a stará se o jejich integritu, ostatní aplikace volají funkce, které nabízí
      \item Mnoho technologií (JAVA RMI, .NET, Webové služby, ...)
    \end{itemize}

  \subsection{Cloud}
    \begin{itemize}
      \item IaaS - Infrasctructure as service
      \item PaaS - Platform as a service
      \item SaaS - Software as a service
    \end{itemize}

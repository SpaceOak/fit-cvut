\section{Configuration management}
  \begin{description}
    \item[Softwarový produkt] Úplný soubor počítačových programů, postupů, související dokumentace, určený pro dodání uživateli.
    \item[Softwarová položka] Jakákoliv identifikovaná část softwarového produktu v průběžném nebo konečném stadiu vývoje.
    \item[Konfigurační řízení] Zajištění plného řízení konfigurace SW produktu a související dokumentace v průběhu životního cyklu.
    \item[Změnové řízení] Součást konfiguračního řízení. Je to řízení rozsahu nad rámec původně domluveného rozsahu.
  \end{description}

  \subsection{Řízení verzí}
  \begin{description}
    \item[SCM/VCM] Verzovací systém (např. \emph{git}, \emph{svn}, ...)
    \item[Konfigurační jednotka] Jeden soubor s konfigurací systému.
  \end{description}

    \begin{itemize}
      \item Identifikace elementů a verzí.
      \item Evidence změn během stádia vývoje.
      \item Umožňuje paralelní práci.
      \item Ozačování verzí pomocí \emph{tagů} a \emph{verzí}.
    \end{itemize}

  \subsection{Změnové řízení}
    \begin{itemize}
      \item Využívá se \emph{issue tracker}, např. Bugzilla.
      \item Měří se čas, který změny zabraly z důvodu nacenění.
    \end{itemize}

  \subsection{Release management}
    \begin{description}
      \item[Integrační platforma] \emph{Cruise control}. Každodenní kontrola zdrojových kódů. Umožňuje
      rychlou zpětnou vazbu v případě problémů.
    \end{description}
    Vytváří se každodenní build, deploy a test.

  \subsection{Průběžná integrace / CI}
    Umožňuje rychlé nasazení, spuštění automatických testů, automatizuje kompilaci, automatizuje code review.
    Vyžaduje, aby tým pracoval s nejaktuálnější možnou verzí projektu.

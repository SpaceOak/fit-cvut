\documentclass[a4paper]{article}

\usepackage[utf8]{inputenc}
\usepackage[IL2]{fontenc}
\usepackage{a4wide}
\usepackage{minted}
\usepackage[czech]{babel}

\renewcommand\listingscaption{Ukázka kódu}
\definecolor{bgcode}{rgb}{0.95,0.95,0.95}

\begin{document}

\section*{Automatizované kompilování \LaTeX{} dokumentů}
Autor: Josef Doležal, květen 2017

\bigskip

\noindent{}V této práci bych rád uvedl návod na automatizované kompilování \LaTeX{} dokumentů uložených v repozitáři na platformě GitHub.\footnote{https://github.com}
Kompilace probíhá pomocí virtuálních strojů na serveru Travis CI.\footnote{https://travis-ci.com}
Pro správné fungování je nutné založit si uživatelský účet na obou platformách.

\subsection*{Propojení účtů}

Po založení účtu na stránce GitHub se lze stejným účtem přihlásit i na Travis CI.
V rozhraní Travis CI se po přihlášení zobrazí seznam dostupných repozitářů.
Z nich vybereme ten, pro který chceme automatickou kompilaci zařídit.
Kompilace se běžně skládá z následujících kroků:

\begin{description}
    \item[before\_install] První fáze před kompilací.
    V této fázi se typicky aktualizují balíčkovací systémy.
    \item[install] V této fázi se stahují veškeré potřebné závislosti nutné ke korektní kompilaci našeho repozitáře.
    \item[script] Fáze samotné kompilace.
    \item[deploy] Poslední fáze spuštěná po úspěšné kompilaci.
    V této fázi se nahrají zkompilované soubory zpět do repozitáře na GitHub.
    Tato fáze se spouští pouze tehdy, je-li vydána nová verze zdrojových kódů v repozitáři.
\end{description}

Nastavením jednotlivých kroků se věnuji následně v části \textit{Konfigurace}.

\subsection*{Konfigurace}

Pro úspěšnou integraci s Travis CI je potřeba nejdříve vytvořit konfigurační soubor \texttt{.travis.yml} v kořenové složce repozitáře.
V tomto souboru se nachází informace o tom, jaký operační systém vyžadujeme, jak repozitář zkompilovat a jaké nástroje k tomu budeme potřebovat.
V kořenovém adresáři repozitáře tedy nyní soubor vytvoříme a otevřeme v libovolném textovém editoru.

\subsubsection*{Výběr operačního systému}

Jako operační systém jsou nabízené Ubuntu 12.04, Ubuntu 14.04 (experimentálně) a macOS.
Strojů s operačním systémem macOS je nabízené velmi omezené množství, proto kompilace na nich zpravidla začne mnohem později než na strojích s Linux OS.
Pro kompilaci \TeX{} dokumentů je navíc tento systém zbytečný.
Z tohoto důvodu budeme uvažovat pouze linuxové distribuce.

Při výběru z dostupných distribucí je potřeba vědět, jaké balíčky a jejich verze v dokumentu vyžadujeme.
Pro Ubuntu 12.04 je dostupná verze \TeX{} Live 2009, ve které je mnoho balíčků zastaralých.
V novějším Ubuntu 14.04 je dostupná verze \TeX{} Live 2013.
I tato verze má některé balíčky (např. \texttt{minted}) zastaralé, novější ale pomocí balíčkovacího systému získat nelze.
V dokumentu dále jako zvolený systém uvažuji Ubuntu 14.04.

Volbu systému provedeme přidáním řádky podle ukázky \ref{yml-os} do konfiguračního souboru.

\begin{listing}[htbp]
\caption{\label{yml-os} Volba OS v konfiguračním souboru}
\begin{minted}[linenos,bgcolor=bgcode]{yaml}
dist: trusty
\end{minted}
\end{listing}

\subsubsection*{Instalace závislostí}

Systém poskytovaný pro kompilaci je primárně určen pro jazyk \texttt{python}.
Pro kompilaci \TeX{} či \LaTeX{} souborů je nutné doinstalovat \TeX{} Live balíček.

Pro instalaci lze využít standarní balíčkovací systém Ubuntu nazvaný \texttt{apt}.
Instalace probíhá ve dvou fázích.
Nejdříve ve fázi \texttt{before\_install} zaktualizujeme balíčkovací systém.
V následující fázi \texttt{install} nainstalujeme veškeré závislosti potřebné ke kompilaci dokumentu.
Ukázka \ref{yml-install} uvádí obsah souboru \texttt{.travis.yml} po přidání konfigurace pro stažení závislostí.

\begin{listing}[htbp]
\caption{\label{yml-install} Instalace závislostí}
\begin{minted}[linenos,bgcolor=bgcode]{yaml}
dist: trusty
before_install:
- sudo apt-get -qq update
install:
- >
  sudo apt-get install -y --no-install-recommends texlive-fonts-recommended
  texlive-latex-extra texlive-fonts-extra dvipng texlive-latex-recommended
  texlive-lang-czechslovak texlive-generic-extra
  lmodern latexmk latex-xcolor
- pip install pygments
\end{minted}
\end{listing}

V této fázi jsou připraveny veškeré závislosti.
Nyní je možné přejít k samotné kompilaci.

\subsubsection*{Kompilace dokumentu}

Po nainstalování závislostí lze dokument zkompilovat obdobným způsobem jako lokálně.
Pro správné vygenerování (nejen) dopředných referencí je dobré využít některého automatizačního nástroje.
Jedním takovým nástrojem je \texttt{latexmk}, který je předinstalovaný z ukázky \ref{yml-install}.
Jakým způsobem budeme dokument kompilovat nakonfigurujeme parametrem \texttt{script}.
V tomto parametru předáme skript, který bude na kompilovacím serveru spuštěn programem \texttt{bash}.
Pro spuštění \texttt{latexmk} vložíme na konec konfiguračního souboru kód z ukázky \ref{yml-script}.

\begin{listing}[htbp]
\caption{\label{yml-script} Spuštění kompilace}
\begin{minted}[linenos,bgcolor=bgcode]{yaml}
script: latexmk -pdf dokument
\end{minted}
\end{listing}

V ukázke \ref{yml-script} je potřeba klíčové slovo \texttt{dokument} nahradit skutečným názvem dokumentu určenému ke kompilaci.

\subsubsection*{Vydání verze dokumentu}

Platforma Travis CI umožňuje nejen průběžnou kompilaci dokumentů, která ověří, že formát není poškozený, ale také jeho vydání.
Vydání dokumentu je přímo navázáno na vlastnost značkování (z anglického \textit{tag}) verzovacího systému \texttt{git}.
Kdykoliv je verzi (\texttt{commit}) zdrojového kódu přidána značka, můžeme nechat Travis CI po úspěšné kompilaci nahrát zkompilovaný dokument k této zpět na GitHub.

Pro zprovoznění této funkce je potřeba nainstalovat program \texttt{travis}.
K instalaci lze využí kód z ukázky \ref{bash-install-travis}.

\begin{listing}[htbp]
\caption{\label{bash-install-travis} Instalace programu travis}
\begin{minted}[linenos,bgcolor=bgcode]{bash}
gem install travis
\end{minted}
\end{listing}

Nyní je nutné repozitář nakonfigurovat pro vydávání verzí.
Ukázka \ref{travis-setup-releases} znázorňuje kód, který konfiguraci zajistí.

\begin{listing}[htbp]
\caption{\label{travis-setup-releases} Instalace programu travis}
\begin{minted}[linenos,bgcolor=bgcode]{bash}
travis setup releases
\end{minted}
\end{listing}

Při prvním spuštění program zahlásí, že se uživatel musí přihlásit ke svému GitHub účtu.
Pokyny k přihlášení se zobrazí v terminálu.
Po úspěšném přihlášení je potřeba kód z ukázky \ref{travis-setup-releases} spustit znovu.
Program konfigurací uživatele provede.
Konfigurace se automaticky zapíše do konfiguračního souboru, ten je ale potřeba dále doupravit.

Sekce s konfigurací vydávání verzí by po úpravě měla mít stejnou strukturu jako kód z ukázky \ref{yml-deploy}.
První dva řádky určují, že budeme pracovat s platformou GitHub a vydávat verze pomocí ní.
Zbylá nastavení jsou podrobněji rozebrána níže.

\begin{listing}[htbp]
\caption{\label{yml-deploy} Nastavení kompilace pro vydání verze}
\begin{minted}[linenos,bgcolor=bgcode]{yaml}
deploy:
  api_key:
    secure: "*****"
  provider: releases
  file: dokument.pdf
  skip_cleanup: true
  on:
    repo: texuser/texrepo
    tags: true
\end{minted}
\end{listing}

Pro správné fungování s libovolným repozitářem je nutné některé údaje podle potřeby nahradit.
Následující část rozebírá tyto nastavení:

\begin{description}
    \item[api\_key] Tato položka obsahuje zašifrovaný autorizační token.
    Ten je využit pro nahrání souborů zpět na GitHub.
    Token je generován programem \texttt{travis}.

    \item[file] Toto nastavení udává, jaký dokument bude do služby GitHub po kompilaci nahrán.
    Soubor je určen relativní cestou ke kořenu repozitáře.

    \item[skip\_cleanup] Travis CI je standardně nastaven, aby po dokončení kompilace soubory smazal.
    To by ale zabránilo jeho následnému odeslání na GitHub.
    Pro správné fungování tedy tato volba musí být nastavena na \texttt{true}.

    \item[on] Toto nastavení udává, za jakých podmínek bude verze dokumentu odeslána na GitHub.
    Skládá se ze dvou podmínek, mezi kterými platí relace \uv{a zároveň}.

    \begin{itemize}
        \item Nastavení \texttt{repo} udává, že verze bude odeslána, jen pokud má repozitář specifický název.
        Název je předán jako hodnota tohoto nastavení a je uveden ve formátu \texttt{uživateské jméno / název repozitáře}.
        \item Druhé nastavení \texttt{tags} přikazuje vydat verzi jen v případě, že kompilace je vyvolána přidáním značky ke zdrojovému kódu.
        Pro správné fungování je potřeba zadata hodnotu \texttt{true}.
    \end{itemize}
\end{description}

V tuto chvíli je repozitář kompletně nakonfigurovaný.
Pomocí verzovacího \texttt{git} nyní můžeme vytvořit \texttt{commit} a ten odeslat na GitHub.
Ve webovém rozhraní v záložce \texttt{Releases} můžeme vydat novou verzi.
Tím se nahraný \texttt{commit} označkuje a spustí se kompilace na Travis CI.
Pokud vše proběhne v pořádku, během několika minut bude u verze dostupný i zkompilovaný dokument.

\end{document}

\section{Přednáška 5 -- MongoDB}

JSON dokumentová databáze.
JSON objekty (doumenty) podobné struktury jsou ukládány do kolekcí.
Každý dokument má unikátní id (interně \texttt{\_id}).
Toto id je unikátní, neměnné a libovolného primitivního typu.
Defaultně je použito ObjectId (speciální BSON typ).

Interně jsou JSON dokumenty uloženy jako binární reprezentace JSON objektu (BSON).
Dokumenty nemají žádné explicitní schéma.

Na vnořené struktury je možné vytvořit tzv. \textit{subdocument}, což je vnořený dokuemnt.
Díky tomu je možné data dostat rychleji (není potřeba explicitně řešit reference).
Ideální pro relace typu \textit{one-to-one} nebo \textit{one-to-many}.

\subsection*{Reference}

Obdnobně jako u relačních databází to jsou související data v \textit{samostatném} dokumentu.
Nemá žádný speciální konstrukt, jsou to id cílových dokumentů.

\subsection*{Dotazování}

Dostupné jsou klasické CRUD operace pomocí \texttt{.insert()}, \texttt{.update()}, \texttt{.remove()}, \texttt{.find()}.

\subsubsection*{Insert}

Vkládat lze jeden nebo kolekce dokumentů s explictním id (musí být unikátní) nebo automaticky generovaným.

\subsubsection*{Update}

Pro aktualizaci je nutné zadat dotaz a aktualizovaná data, případně konfiguraci.
Aktualizovaný je vždy nejvíše jeden objekt, to lze změnit konfigurací \texttt{\{multi: true\}}.
Pokud aktualizace používá \textbf{jen} aktualizační operátory (\texttt{\$set}, \texttt{\$unset}, \dots) je dokument aktualizován, jinak je celý nahrazen.

Chceme-li aktualizovat existující dokument nebo vytvořit nový pokud ještě neexistoval, můžeme použít \textit{upsert} pomocí konfigurace \texttt{\{upsert: true\}}.

\subsubsection*{Remove}

Odstraňuje všechny záznamy odpovídající dotazu pokud není zadána konfigurace \texttt{\{justOne: true\}}.

\subsubsection*{Find}

Vrací dokumenty odpovídající zadané query, projekci je možné zadat jako druhý parametr.

\section{Přednáška 2 -- SQL optimalizace}

Složitějíší SQL příkazy se skládají ze základních relačních operací (selekce, projekce, spojení, řazení, \dots).
Vyhodnocení dotazu je možné popsat pomocí stromu.
Tento strom se nazývá \textbf{exekuční plán} a pro jeden dotaz jich může existovat více.
Optimalizací SQL se rozumí nalezení stromu s nejmenší cenou.

\subsection*{Optimalizace}

\begin{description}
    \item[Cena] Počet I/O bloků, které je potřeba načíst do paměti.
\end{description}

\subsection*{(Heap) tabulky}

\subsubsection*{Statistiky}

\begin{description}
    \item[nR] Počet n-tic (řádků) v relaci R.
    \item[\(V(A, R)\)] Počet různých hodnot A v relaci R.
    \item[pR] Počet stránek potřebných k uložení relace R.
\end{description}

\subsection*{B-stromy}

\subsubsection*{Statistiky}

\begin{description}
    \item[\(f(A, R)\)] Průměrný počet následníků uzlu.
    \item[\(I(A, R)\)] Hloubka indexového stromu.
    \item[\(p(A, R)\)] Počet listů indexového stromu.
\end{description}

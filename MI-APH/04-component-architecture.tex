\section{Přednáška 4 -- Komponentová architektura pt. 1}

\subsubsection*{Game Model}

Nejmenší podmnožina hry, která určuje veškerý její stav.
Může být rozdělený na více částí (core, view, síť, \dots).
S každým updatem je aktuální stav game modelu vykreslen na obrazovku.

\subsubsection*{View Model}

Prezentovaný na obrazovce po zpracování všech ostatních modelů.

\subsubsection*{Networking Model}

Stará se o stav, který musí být synchronizovaný na jiná zařízení.

\subsubsection*{Core Model}

Klíčové entity a mechanika hry (pro složité hry těžké určit).
Měl by být navržen tak, aby mohl být jednoduše oddělen od ostatních částí a testovatelný nezávisle.

\subsubsection*{Data Model}

Popisuje stav v každém okamžiku hry.
Blízce spjatý se síťovým modelem (pokud existuje).

\subsubsection*{Physics Model}

Entity, které jsou jakkoliv ovlivňovány fyzikální interakcí.

\subsubsection*{AI Model}

Všechna data zpracovávaná AI (navigační mapy, behaviorální stromy, \dots).

\subsection*{Objektový přístup}

\subsubsection*{All-in-one object pattern}

Vzor používaný v 90. letech.
Jedna třída obsahuje chování všech herních objektů.
Vnáší do hry spoustu kontrolních mechanizmů (if, switch).

Objektový návrh nám díky dědičnosti umožňuje jednoduše určovat hierarchii.
Pro komplexní hry ale přináší spoustu závislostí, herní vztahy objektů navíc nemusí být acyklický graf.

Tato architektura je jednoduchá a rychlá na vývoj, ale neškáluje a je těžko udržitelná.

\subsection*{Komponentový přístup}

Objektový návrh (především dědičnost) nedokážou vždy vystihnout podstatu závislostí herních objektů.
Komponentová architektura je založená na kompozici.

\subsubsection*{Komponenta}

Jednotka kompozice se specifikovanými rozhraními a explictními závislostmi.
Jsou to v zásadě \textit{plug-and-play} pro herní objekty.

Může být jednoduše integrována do \textit{screne graph}.

\subsection*{Architektury}

\begin{description}
    \item[Monolithic] Tradiční OOP přístup. Má problémy s hlubokými a širokými hierarchiemi.
    \item[Object-centric] Každý objek je reprezentovaný v runtimu jako instance třídy.
    \item[Property-centric] Objekty jsou reprezentovány jen pomocí unikátního ID (sring, int). Komponenty jsou uloženy v paměti za sebou (např. v poli).
\end{description}

\subsubsection*{ECS(A) Pattern}

Herní objekt je jen kontejner na data a logiku.
\textbf{E}ntity-\textbf{C}omponent-\textbf{S}ystem (\textbf{A}ttribute)
Veškeré chování konkrétního objektu je zcela určeno shromážděním jeho komponent (a atributů).

\img{ecsa.png}{ECS(A) Pattern}{ecsa}{0.4}

Každý engine založený na ECS(A) má vlastní implementaci, neexisute žádná striktní specifikace.

\img{ecsa-example.png}{ECS(A) ukázka}{ecsa-example}{0.75}

\subsubsection*{Game Model}

Pro implementaci herního modelu máme více možností jak herní model implementovat.

\begin{enumerate}
    \item Celý \textit{scene graf} uložit jako herní model. Lepší pro komplexní hry. Celý model je zadrátován do scény natvrdo.
    \item Model jako atribut herního \textit{root} elementu. Lepší pro malé (bez fyzikální enginu). Horší performance, protože je potřeba model synchronizovat se \textit{scene graph}.
    \item Kombinace předchozích. Potřeba synchronizace.
\end{enumerate}

\subsubsection*{Pojmy}

\begin{description}
    \item[Entity] (\textit{GameObject}) Reprezentuje jednu konkrétní entitu, typicky uzel \textit{scene grafu}.
    \item[Attribute] Náhrada za členskou proměnnou, obsahuje \textbf{data}
    \item[Component] Entita obsahující \textit{kód} funkcionality. 
    \item[Behavior] Použitelná komponenta, která můvze být připojena k objektu.
    \item[(Sub)system] Globální komponenta sdílená různými objekty (Renderer, Audio).
    \item[Message] Náhrada za volání metod, slouží ke \textbf{komunikaci} mezi komponentami.
    \item[Event] Typ zprávy -- zastupuje událost, která již nastala (\textit{broadcast} události).
    \item[Command] Typ zprávy -- zastupuje akci, která má nastat (\textit{unicast}).
    \item[Process (Job)] Práce, která zabere více než jeden frame k vykonání.
\end{description}

Komponentová architektura je škálovatelná, datově orientovaná s jednoduchým recyklováním komponent/chování.
Je ale dynamicky typovaná, heirarchie komponent není sama o sobě znovu použitelná. závislosti musí být natvrdo pospojované a systém zpráv se těžko předělává jakmile je jednou hotový.

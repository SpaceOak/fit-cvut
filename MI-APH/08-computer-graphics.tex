\section{Přednáška 8 -- Počítačová grafika}

\subsection*{Prostory}

\subsubsection*{Model space}

Prostor umístěný typicky na středový bod (střed hmoty).
Osy jsou zarovnány v přirozeném směru modelu.

\subsubsection*{World space}

Fixní souřadnicový prostor, vůči kterému jsou pozice, rotace a škálování ostatních objektů vyjádřeny.

\subsubsection*{View/Camera space}

Souřadnicový systém vzhledem ke kameře.
Počátek systému je umístěn vzhledem k ohnisku kamery (?).

\subsubsection*{Clip space}

Čtvercový hranol se souřadnicemi v intervalu \([-1, 1]\).

\subsubsection*{View/Screen space}

Oblast obrazovky využitá k zobrazení části výsedného obrazu (?).

\img{spaces.png}{Vizualizace prostorů}{spaces}{0.7}

\subsubsection*{View volume}

Region prostoru, který je zabíraný kamerou.

\subsubsection*{Interpolace}

Metoda výpočtu/nalezení bodu(ů) v daném intervalu.
Využívá se ke změně velikosti obrázků, animacím (přechod mezi transformacemi) nebo ke zpracování zvuku (interpolace vzorků).

\subsubsection*{Lineární interpolace}

\[
    y = y_1 + (y_2 - y_1) \cdot \frac{x-x_1}{x_2-x_1}
\]

\img{linear-interpolation.png}{Lineární interpolace}{linear-interpolation}{0.3}

\subsection*{Vykreslování}

\medskip

\subsubsection*{Pojmy}

\begin{description}
    \item[Vertex] (\textit{Vrchol}) Primárně bod ve 3D prostoru se souřadnicemi. Má atributy jako: pozicionální vekor, normála, barva, \dots
    \item[Fragment] Vzorkovaný segment rastrovaného primitiva (?). Velikost závisí na metodě vzorkování.
    \item[Texture] Kus bitmapy aplikovaný na model.
\end{description}

\subsubsection*{Shaders}

Programy, které běží na grafické kartě aby využívali množství speciálních funkcí (osvětlení, effeky, post-processing, fyzika, \dots).

\begin{description}
    \item[Vertex shader] Vstupem je vrchol, výstupem je transformovaný vrchol.
    \item[Geometry shader] Vstupem je n-vrcholové těleso, výstupem je žádné nebo více primitiv.
    \item[Tessellation shader] Vstupem je primitivum, výstupem je rozdělené primitivum.
    \item[Pixel/fragment shader] Vstupem je fragent, výstupem je brava, hloubka a šablona.
    \item[Compute shader] Shader běžící mimo renderovací smyčku, využívané pro masivní paralelní výpočty.
\end{description}

\subsection*{Vlastnosti vykreslování}

\medskip

\subsubsection*{Textury}

Kus bitmapy, který je aplikován na objekt.
Skládá se z \textbf{texelů} (individuální element textury).

\subsection*{Sprites}

Sprite je jeden grafický obrázek začleněný do scény.
Může být renderován jako \textit{squad} (skupina obrázků) sestavěná z dvou trojúhelníků nebo hajo jeden bod pomocí \textit{geometry shaderu}.
Může tak být vyjresleno tisíce spritů během jednoho volání vykreslovací funkce.

Sprity jsou seskopovány do \textbf{spritesheetů}, které shlukují skupinu souvisejících spritů (např. různé stavy objektu).

\textbf{Sprite atlas} je jedna textura obsahující jeden nebo více spritesheetů.

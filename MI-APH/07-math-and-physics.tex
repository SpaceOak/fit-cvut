\section*{Přednáška 7 -- Herní matematika a fyzika}

\subsection*{Dynamika}

\medskip

\subsubsection*{Body a vektory}

\medskip

\begin{description}
    \item[Vektor] Veličina, která má velikost i směr.
    \item[Velikost] (\textit{Magnitude}) Skalár reprezentující délku vektoru (\(|a| = \sqrt{a_x^2 + a_y^2 + a_z^2}\)).
    \item[Normálový vektor] Vektor kolmý k povrchu.
    \item[Skalární součin] \(a \cdot b = a_xb_x + a_yb_y + a_zb_z\) případně \(a \cdot b = |a| \cdot |b| \cdot \cos{\theta}\).
\end{description}

\subsubsection*{Verletova integrace}

Numerická metoda výpočtu používaná k integrování rovnic pohybu.
Používá se např. při výpočtu balistické křivky (konkrétně Velocity Verlet, který má oproti Eulerovské metodě výhodu při výpočtech s více objekty).

\subsection*{Steering behaviors}

Množina algoritmů a principů pro pohyb autonomních agentů realistickým způsobem.
Agentem nazýváme systém umístěný v prostředí.

Rozlišujeme více typů těchto chování.

\begin{description}
    \item[Seek] Síla směrující agenta k cílové pozici.
    \item[Flee] Protiklad \textit{seek}.
    \item[Arrive] Pohybuje agentem tak, že ho zpomaluje při přiblížení k cíli. Přidává se parametr \texttt{slowingRadius}.
    \item[Pursuit] Agent sleduje cíl a predikuje kde se v budoucnu bude nacházet.
    \item[Evade] Agent uniká od předpokládané budoucí pozice cíle.
    \item[Wander] Vytváří sílu, která budí dojem \textit{náhodné procházky}.
    \item[Path follow] Agent sleduje množinu předem danných bodů (konečný pomocí \textit{arrive}, ostatní pomocí \textit{seek}).
    \item[Obstacle avoidance] Vede agenta tak, aby se vyhýbal překážkám.
    \item[Offset pursuit] Drží agenta v určité pozici oproti \textit{leaderovi}. Vhodné např. pro bitevní pozice. 
\end{description}

\subsection*{Fyzikální enginy}

Systém, který simuluje fyzikální chování.
Simulace probíhá v reálném čase.

\img{physics-engine-loop.png}{Smyčka fyzikálního enginu}{physics-engine-loop}{0.2}

\subsubsection*{Pojmy}

\medskip

\begin{description}
    \item[Rigid body] Idealizované, nedeformovatelné objekty. Dělí se do tří kategorií:
        \begin{itemize}
            \item Physics driven -- řízený pouze simulací,
            \item Game driven -- pohybující se nepřirozeným stylem (např. animace),
            \item Fixed -- objekty určené pouze k detekci kolizí.
        \end{itemize}
    \item[Constraint] Spojuje \textit{bodies} tak, aby simuloval interakce mezi nimi (řetězy, provazy, kola, \dots).
\end{description}

\subsubsection*{Constraints}

Rozlišujeme mnoho typů \textit{constraints}, např.:

\begin{description}
    \item[Revolute] Bod, kde se \textit{bodies} otáčí (kola, řetězy, otočné dveře, \dots).
    \item[Prismatic] Objekt se pohybuje jen pod určitým úhlem (výtah, posuvné dveře, \dots).
    \item[Cone-twist] Přidává cones \& twist limits (?).
\end{description}

\img{constraints.png}{Příklady constraints}{constraints}{0.6}

\subsubsection*{Primitiva}

Rozlišujeme základní primitiva, ze kterých se skládají složitější objekty:

\begin{description}
    \item[Sphere] Kruh, koule -- definováno 4 bodami (\(x, y, z + \textrm{radius}\))
    \item[Capsule] Kružnice se čtvercem, případně kvádr s koulí -- definováno dvěma body a průměrem.
    \item[AABB] \textit{Axis-aligned bouding box}. Kvádr, jehož hrany jsou rovnoběžné s osami. Efektivní na hledáí průniků.
    \item[OBB] \textit{Oriented bounding box}.
\end{description}

\img{primitives-example.png}{Ukázka primitiv}{primitives-example}{0.8}

\subsubsection*{Detekce kolizí}

\medskip

\subsubsection*{Separating axis theorem (SAT)}

Detekce kolize dvou rovnoběžníků se dělá na základě projekcí na osy.
Pro oba objekty vezmeme jejich osy s nimi rovnoběžné.
Pokud existuje alespoň jedna osa, kde se projekce neprotínají, pak se ani objekty neprotínají.

\subsubsection*{Collision query}

Operace používané k zodpovězení otázek jako \uv{\textit{Který objekt kulka trefí jako první?}}, \uv{\textit{Jaké objekty se vyskytují v daném poloměru?}}

\subsubsection*{Reakce na kolize}

\begin{description}
    \item[Explosion] Přidání energie systému \textit{rigidních těles}.
    \item[Friction] (\textit{Tření}) Odebírání energie systému \textit{rigidních těles}.
    \item[Restitution] Velikost odrazu po střetu s něčím (?).
    \item[Friction] (\textit{znovu?}) Síla vznikající mezi dvěma tělesy, které jsou v kontaktu delší dobu.
    \begin{itemize}
        \item \textit{Static} -- Oproti povrchu.
        \item \textit{Dynamic} -- Objekty se pohybují relativně k sobě.
        \item \textit{Rolling} -- Bod dotyku mezi povrchem a kolem.
    \end{itemize}
\end{description}

\subsubsection*{Částicové systémy}

Kolekce bodů hmoty, které odpovídají nějakému konkrétnímu fyzikálnímu zákonu.
Hojně využívá \textit{FlyWeight pattern}.
Mají monho atributů (zrychlení, orientace, pozice, hmota, velikost, \dots).
Např. oheň, voda, déšť, kouř, mlha, \dots

\subsection*{Prostor}

\medskip

\subsubsection*{Dělení prostoru}

Struktury udržující objekty podle jejich pozice.
Používají se k hledání sousedů, analyzování tvarů atd.

Implementováno např. pomocí \textit{Grid}, \textit{Quad-tree} (2D), \textit{Octree} (3D), binárního vyhledávacího stromu, \dots

\begin{description}
    \item[Grid] Každá buňka má seznam prvků uvnitř, ideální pro velké množství objektů stejné velikosti.
    \item[Quad-tree] Hierarchická struktura, kde každý uzel má právě 4 následníky. Objekty jsou umístěny v listech. Dobré pro malé množství objektů různé velikosti. 
\end{description}

\subsubsection*{Rotace}

\begin{description}
    \item[Eulerovy úhly] Jednoduchá reprezentace (3 čísla -- 3 osy), ale trpí na \textit{gimbal lock} -- ztrátu možnosti otáčení se po jedné ose pře natočení o \(90^\circ\).
    \item[Matice] Netrpí na \textit{gimbal lock}, ale jsou neintuitivní, zaberou více místa a nelze je jednoduše interpolovat.
    \item[Axis + angle] Pomocí osy a úhlu (\([a_x, a_y, a_z, \theta]\)). Intuitivní a kompaktní, ale nelze použít přímo na body a vektory.
    \item[Quaternions] (\textit{Čtveřice}) Podobné jako \textit{axis + angle}, ale osa rotace je škálovaná hodnotou \texttt{sin} hodnoty poloviny úhlu rotace. Jednoduché na interpolaci, lze použít přímo.
\end{description}

\subsubsection*{Quaternions}

Umožňují nahradit veškeré násobení matic a lze je po výpočtu na matice opět převést.

\[
    Q = (\cos{\frac{\varphi}{2}}, x \cdot \sin{\frac{\varphi}{2}}, y \cdot \sin{\frac{\varphi}{2}}, z \cdot \sin{\frac{\varphi}{2}})
\]




% Quaternions: 𝑄 = (cos 𝜑 , 𝑥 ⋅ sin 𝜑 , 𝑦 ⋅ sin 𝜑 , 𝑧 ⋅ sin 𝜑) 2222

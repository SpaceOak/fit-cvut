\section{Přednáška 1 -- Úvod do světa her}

\subsection*{Počítačová hra}

\medskip

\begin{description}
    \item[Počítačový program] Sada instrukcí spouštěná počítačem.
    \item[Aplikace] Program pomáhající uživateli vykonal úkol.
    \item[Hra] Počítačem kontrolovaná hra kde uživatelé interagují s objekty na obrazovce aby se zabavili. Nejedná se o aplikaci. Jedná se real-time simulátory.
\end{description}

\subsection*{Základní pojmy}

\medskip

\begin{description}
    \item[Emergence] (\textit{Vznik}) Odkazuje se na fakt, že chování je výsledkem složitého a dynamického systému pravidel.
    \item[Progression] (\textit{Postup}) Struktury ve hře, u kterých designer předem definoval možné herní stavy, např. pomocí levelů.
    \item[Gameplay] (\textit{Hratelnost}) Vlastnost hry definována jejími pravidly(?).
    \item[Mechanics] (\textit{Mechanika}) Množina pravidel definující chování jednoho herního prvku.
    \item[System] Interagující skupina herních prvků tvořící ucelený celek.
    \item[Level] (\textit{Úroveň}) Struktura ve hře určující s jakými výzvami se hráč setká.
    \item[Simulation] Reprezentace zdrojového systému pomocí jednoduššího systému, který zdrojový systém napodobuje.
\end{description}

\subsection*{Základní herní elementy}

\begin{itemize}
    \item (Středně viditelné) Mechanika,
    \item (Středně viditelné) Příběh (posloupnost událostí rozvíjející hru),
    \item (Výrazně viditelné) Estetika (vzhled hry, zvuky, pocity, \dots),
    \item (Méně viditelné) Technologie (veškeré součásti a interakce tvořící hru).
\end{itemize}

\subsection*{Herní mechanika}

\begin{itemize}
    \item Prostor. Množiny prostorů a jejich vzájemné propojení.
    \item Objekty a atributy. Viditelné objekty, se kterými může být manipulováno.
    \item Akce. Možnost objektu \textit{jednat}.
    \item Pravidla. Definují omezení, následky a cíle.
\end{itemize}

\subsection*{Herní enginy}

Nástroj (nejen kus software) pro rychlý vývoj interaktivních systémů poskytující okamžitou zpětnou vazbu.

\subsubsection*{ID Tech}

Rodina herních enginů od společnosti id Software.
Prvním herním enginem byl dd Tech 0.
Následníkem byl id Tech 1 (1996) s real-time 3D grafikou (např. Quake).

\subsubsection*{Unreal Engine}

Vytvořen v roce 1998 firmou Epic Games na FPS hry.
První hra v tomto enginu byla \textit{Unreal}.
Obsahoval detekování kolizí, HW i SW renderování, \dots

\subsubsection*{Další enginy}

Existují i další enginy, např CryEngine nebo general purpose engine Unity.

\documentclass[11pt,a4paper,czech]{article}

\usepackage[utf8]{inputenc}
\usepackage[IL2]{fontenc}
\usepackage{a4wide}
\usepackage[czech]{babel}


\title{FI-FIL}
\author{Josef Doležal}

\begin{document}

\maketitle

\newpage

\section{Konec Archaické doby -- raní předsokratici (konec 6. stol. př. n. l.)}

\subsection{Thalés z Mílétu}

    \begin{itemize}
        \item Milétská škola,
        \item první filosof, astronom, geometr, ukázal jako první Řekům zkoumání přírody,
        \item předpověděl zatmění slunce, délku roku, thaletova věta, kružnice, dálkoměr, dělení klenby na 5 pásem,
        \item názor, že plocha Země plave na vodě, oceánu,
        \item táže se, co je základem všeho -- nespokojil se jen s mýtem,
        \item hledání principu věcí -- \textit{arché} -- volí vodu
    \end{itemize}

    \begin{description}
        \item[Redukcionismus] Není rozdíl mezi živým a neživým, vše je živé.
    \end{description}

\subsection{Anaximandros z Mílétu}

\begin{itemize}
    \item nejstarší filosofický text,
    \item slavný díky konstrukci mapy,
    \item zmiňuje \textit{diference} -- protiklady (mokré x suché, teplé x studené),
    \item ve spojení s diferencemi popisuje cykly ve světě (roční obbdobí, počasí)
\end{itemize}

\subsection{Anaximandros z Mílétu}

\begin{itemize}
    \item astronom, především meteorolog,
    \item počátek všeho vidí ve vzduchu (rozpínání x smrťování, změny skupenství),
    \item plochá země se vznáší ve vzduchu jako list
\end{itemize}

\subsection{Pýthagorovci}

    \begin{itemize}
        \item hledači \textit{harmonie}, vidí ji v číslech,
        \item matematizace běžného světa (dvojice, trojúhelník (milostný, geometrický)),
        \item náboženské hnutí, \textit{sekta}, přesto zkoumají vesmír (astrologie) a svět kolem sebe (matematika, lékařství),
        \item škola se rozdělovala na vnitřní kruh (učedníci, zabývající se zkoumáním) a vnější (posluchači, pouze aplikovali pravidla spolku na svůj život, znali výsledky některých zkoumání),
    \end{itemize}

    \subsubsection{Pýthagoras ze Samu}

    \begin{itemize}
        \item většina legend o něm vzniká až dlouho po jeho smrti,
        \item zakladatel pýthagorské školy,
        \item Pythagorova věta - krize matematiky -> iracionální čísla,
        \item Země kulatá, obydlená všude -> protinožci -> gravitace,
        \item Hvězdy a planety jsou bozi,
        \item Odpoutání duše od těla
    \end{itemize}

    \subsubsection{Filolálos z Krotónu}

    \begin{itemize}
        \item termín KOSMOS,
        \item ne-geocentrický systém (10 planet, Anti-země), centrální oheň vesmíru (jádro Galaxie)
    \end{itemize}

\subsection{Herakleitos z Efesu}

\begin{itemize}
    \item učení o \textit{logu} -- řeči, rozum věcí,
    \item \uv{Velká učennost neučí rozumu}
\end{itemize}

\subsection{Esejská škola}

    \begin{itemize}
        \item zabývá se řešením otázek obecně (metafyzika),
        \item přikládají význam logickému myšlení
        \item 
    \end{itemize}

    \subsubsection{Xenofanés z Kolofónu}

    \begin{itemize}
        \item existuje jeden Bůh (\uv{metafyzická theologie})
    \end{itemize}

    \subsubsection{Zénón Elejský}
    
    \begin{itemize}
        \item důkaz sporem,
        \item koncept nekonečna, teorie množin,
        \item paradoxy: Achiles a želva (nekonečné řady), letící šíp stojí (později zavedena limita)
    \end{itemize}

\subsection{Atomisté}

    \begin{itemize}
        \item vše se skládá z malých nedělitelných částeček, ty se skládají dohromady,
        \item vše se děje díky osudu (ten je sílou nutnosti)
    \end{itemize}

    \subsubsection{Démokritos}

    \begin{itemize}
        \item učí, že svět má podobu koule,
        \item svět ani hvězdy nemají duši, nejsou božské (kontrast s Pýthagorejci),
        \item lidé by si měli vážit více duše než těla -- dokonalost duše napravuje chyby těla
    \end{itemize}

\subsection{Absolutní determinismus}

    \subsubsection{Epikúros}
    
    \begin{itemize}
        \item člověk je svobodný -- může dělat vše co chce,
        \item člově je zároveň determinovaný -- musíme dělat to, co chceme
    \end{itemize}

    \subsubsection{Stoikové}

    \begin{itemize}
        \item zabývají se vztahem mezi kosmickým determinismem a lidskou svobodou,
        \item cesta ke svobodě je dána žitím s přírodou
    \end{itemize}

\subsection{Sofisté}

    \begin{itemize}
        \item učitelé moudrosti, zlatý věk Athén,
        \item vzhledem k poptávce po vzdělání byli vítáni,
        \item moudrostí se zabývali pro výdělek,
        \item dovádějí starověkou filosofii ke krizi
    \end{itemize}

    \subsubsection{Protágoras z Abdér}

    \begin{itemize}
        \item řečník, rozděluje řeč na čtyři druhy: prosba, otázka, odpověď a rozkaz,
        \item liší rody jmen: mužská, ženská a věcná,
        \item kvůli své nejistotě o existenci bohů nucen emigrovat na Sicílii,
    \end{itemize}

\section{Filosofie klasické doby}

    \begin{itemize}
        \item odvracejí úvahy o povaze světa k otázkám o člověku a společnosti,
    \end{itemize}

    \subsection{Sókratés}

    \begin{itemize}
        \item zpochybňuje běžné pravdy,
        \item cílí na jasné definování pojmů
        \item rozmlouval s posluchači o filosofických otázkách,
        \item rozmlouvá dialogem -- rozmlouval s posluchači a logickými argumenty je dováděl ke sporu
        \item cílem filosofie je péče o duši -- \uv{Zlo není záměrné, je to nevědění.}
        \item Sókratovské školy:
            \begin{itemize}
                \item Magerská
                \item Kynická
                \item Hédonická
            \end{itemize}
        \item v roce 422 př. n. l. jeho žáci způsobí další fázi Pelopénské války, kde Athény prohrají
        \item Soud se Sókratem podezřelým z bezbožnosti
        \item poslední den líčen v dialogu Faidón -- přesvědčuje své učedníky, že nemají mít strach ze smrti
    \end{itemize}

    \subsubsection{Megarská škola -- Euklidés}

    \begin{itemize}
        \item Epimenidův paradox -- \uv{Jeden Kréťan tvrdí, že všichni Kréťané lžou. Lže tedy také?}
    \end{itemize}

    \subsection{Platón}

    \begin{itemize}
        \item původním jménem Aristokles
        \item zakladatel Athénské Akadémie -- \uv{Nevstupuj nikdo neznalý uvažování na geometrický způsob}
        \item Sókratův žák
        \item psal dialogy, formou byly dialogy se Sókratem
        \item Dílo Obrana Sókratova, hlavní pramen informací o soudu se Sókratem
        \item Sókratův proces a smrt jsou tématem tří nejznámějších dialogů, odsuzuje Aristofána za způsobení jeho smrti
        \item učení o státu -- \uv{Nejlepší s nejlepšími, nejhorší s nejhoršími.}
    \end{itemize}

    \subsection{Aristotelés}

    \begin{itemize}
        \item nejvýznamější Platónův žák,
        \item učitel Alexandra Makedonského (Velikého),
        \item po smrti A. V. obdobně jako Sókratés obviněn z bezbožnosti, ale před soudem prchá,
        \item rozsáhlé encyklopedické dílo pokládá základ mnoha vědám,
        \item vydává mnoho knih -- Metafyzika, O nebi, O vzniku a zániku, O zkoumání živočichů, Rétorika, Politika, \dots
        \item struktura pojednání:
            \begin{itemize}
                \item předmět zkoumání a záporné otázky
                \item názory předchůdců
                \item kritická analýza těchto názorů
                \item vlastní názory a řešení problémů
            \end{itemize}
        \item mluví o vládě rozumu nad pudy,
        \item nutnost zdokonalovat rozum
    \end{itemize}

    \subsubsection{Kosmos}

    \begin{itemize}
        \item Nebeská tělesa tvořena éterem
        \item sféry mají tělesnou existenci, sféry pohybují 55 hybateli (jména planet podle hybatelů)
    \end{itemize}

\end{document}
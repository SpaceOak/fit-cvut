\documentclass[czech]{article}

\usepackage[utf8]{inputenc}
\usepackage[IL2]{fontenc}
\usepackage[czech]{babel}
\usepackage[a4paper,textheight=674pt]{geometry}
\usepackage{hyperref}
\usepackage{graphicx}
\usepackage{amsfonts}
\usepackage{amsmath}
\usepackage{float}

\newcommand{\lecture}[1]{
    \newpage
    \input{#1}
}

\newcommand{\pspace}{
    $(\Omega, \mathcal{F}, P)$
}

\newcommand{\img}[2]{
    \begin{figure}[H]
        \centering
        \includegraphics[width=0.6\textwidth]{MI-SPI/assets/#1}
        \caption{#2}
    \end{figure}
}

\author{Josef Doležal}
\date{\today}
\title{MI-SPI\\Statistika pro informatiku}

\begin{document}

\maketitle
\newpage

\tableofcontents

\lecture{MI-SPI/01-common-terms}
\lecture{MI-SPI/02-random-variable}
\lecture{MI-SPI/03-random-vectors}
\lecture{MI-SPI/05-entropy}
\lecture{MI-SPI/06-code-theory}
\lecture{MI-SPI/07-differencial-entropy}
\lecture{MI-SPI/08-limits}
\lecture{MI-SPI/09-statistics}
\lecture{MI-SPI/10-hypothesis}
\lecture{MI-SPI/11-hypothesis-2}
\lecture{MI-SPI/13-histograms}
\lecture{MI-SPI/14-random-processes}
\lecture{MI-SPI/15-random-processes-construction}

\end{document}
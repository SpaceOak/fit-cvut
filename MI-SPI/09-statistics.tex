\section{Přednáška 9 -- Statistika}

Narozdíl od pravděpodobnosti postupuje naopak -- na základě skutečných výsledků vybírá správný model, odhaduje parametry, \dots.

\begin{description}
    \item[Náhodný výběr rozdělení $F$:] $n$-tice i.i.d. veličin $X_1, \dots$ s distribuční funkcí $F$,
    \item[Realizace náhodného výběru:] $n$-tice konkrétních pozorování
    \item[Statistika]: funkce náhodného výběru
\end{description}

\subsection{Bodové odhady}

Z naměřených hodnot můžeme odhadnout aproximaci parametru tvaru rozdělení -- \textit{bodový odhad}.

\subsubsection*{Druhy bodových odhadů}

\begin{itemize}
    \item Výběrový průměr (pro stř. hodnotu) $\overline{X}_n = \frac{1}{n}\sum_{i=1}^{n}{X_i}$,
    \item Výběrový rozpty (odhad rozptylu) $s_{n}^2 = s_{X}^2 = \frac{1}{n-1}{\sum_{i=1}^{n}{(X_i - \overline{X}_n)^2}}$,
    \item Výběrová směrodatná odchylka (odhad odchylky) $s_n = \sqrt{s_{n}^2}$,
    \item Výběrová kovariace $s_{X,Y} = \frac{1}{n-1}\sum_{i=1}^{n}{(X_i - \overline{X}_n)(Y_i - \overline{Y}_n)}$,
    \item Výběrový korelační koeficient $r_{X,Y} = r = \frac{s_{X,Y}}{s_X s_Y}$ ($s_X$ je odmocnina z výběrového rozptylu)
\end{itemize}

\subsection{Vlastnosti bodových odhadů}

\subsubsection*{Nestranný}

Odhad $\hat{\theta}_n$ parametru $\theta$ je nestranný, jestliže pro každé $\theta \in \Theta$ platí $E\hat{\theta}_n = \theta$.

Nestrannost znamená, že odhad není zatížen systematickou chybou (hodnotu systematicky nezvětšuje ani nezmenšuje).

Pro binomické, Poissonovo a exponencionální e výběrový průměr nejlepším nestranným odhadem střední hodnoty.
Pro normální je výběrový rozpty nejlepším odhadem střední hodnoty.

\subsubsection*{Konzistentní}

Odhad $\hat{\theta}_n$ parametru $\theta$ je konzistentní, jestliže pro každé $\theta \in \Theta$ platí $\hat{\theta}_n \rightarrow^{P}\theta$ pro $n$ jdoucí do nekonečna.

Konzistence znamená, že volbou $n$ lze učinit chybu dostatečně malou.

\subsection{Konstrukce}

\subsubsection*{Metoda momentů}

Parametry se vyjádří z odhadů momentů -- ty se odhadují odhadnout výběrovými momenty.

\subsubsection*{Metoda maximální věrohodnosti}

Odhad je argumentem maxima věrohodnostní funkce.
Často se maximalizuje lofaritmus věrohodnostní funkce.

\subsection{Intervaly spolehlivosti}

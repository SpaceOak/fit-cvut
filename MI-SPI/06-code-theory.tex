\section{Přednáška 6 -- Teorie kódování}

Zabývá se problémem jak zakódovat zdrojovou zprávu (informaci) do posloupnosti symbolů tak, aby byl přenos co nejefektivnější (největší komprese, nejmenší náchylnost k chybám).

K dispozici je abecda $\mathcal{D}$ obsahující $D$ symbolů -- $D$-ární abeceda.
Dále je zpráva $x_1, \dots, x_n$ konečné délky znaků z množiny $X$.

\subsection{Kód náhodné veličiny}

\begin{description}
    \item[Kód:] Zobrazení $C: \mathcal{X} -> \mathcal{D}^*$ do množiny konečných řetězců symbolů abecedy
    \item[Kódové slovo:] Obraz $C(x)$ prvku $x \in X$
    \item[Délka kódového slova]: $l(x)$
\end{description}

\subsection{Střední délka kódu}

$$
    L(C) = \sum_{x\in X}{l(x)p(x)}
$$

\subsection{Typy kódů}

\subsubsection*{Nesingulární}

Pokud je $C$ prosté (tedy: $x\neq x' \Rightarrow C(x) \neq C(x')$).
Tato vlastnost je dostatečná pro dékodování (dekonstrukci) slov.

\subsubsection*{Jednoznačně dekódovatelný}

Rozšíření $C^*$ kódu $C$ je zobrazení množiny $X^*$ do množiny $C^*$ definované jako

$$
    C^*(x_1x_2\cdots x_n) = C(x_1)C(x_2)\cdots C(x_n)
$$

kde $C(x_1)C(x_2)\dots$ znamená zápis jednotlivných kódových slov za sebe.

Kód $C$ je jednoznačně dekódovatelný, pokud je $C^*$ nesingulární.
Je tedy možné jednoznačně dekódovat každou zprávu.

\subsubsection*{Instantní (prefixový) kód}

Kód, kde žádné kódové slovo není prefixem jiného kódového slova.

\subsubsection*{Hierarchie kódů}

\img{06-code-hierarchy.png}{Hierarchie kódů}

\subsection{Kraftova nerovnost}

Pro libovolný instantní kód nad $D$-ární abecedou délky kódových slov $l_1,\dots$, splnit nerovnost

$$
    \sum_{i}{D^{-l_i} \leq 1}
$$

Podle věty McMillana platí to samé i pro jednoznačně dekódovatelný kód (tedy nenabízí žádné další možnosti délek oproti instantnímu).

\subsection{Dolní mez délky instatního kódu}

Střední délka $L(C)$ je $L(C) \geq H_D(X)$.
Rovnost nastane právě když $D^{-l_i} = p_i$.

\subsection{Optimální kód}

Pro optimální instatní kód platí

$$
    H_D(X) \leq L(C^*) < H_D(X) + 1
$$

\subsection{Huffmanovo kódování}

Seřazení slov podle pravděpodobnosti (dolu ty s nižší).
Spojí se dvě s nejnižší a sečte se jejich pravděpodobost.
Takto se spojují všechny dokud nezůstane 1.
O tohotu uzlu se rozmisťují na větve 0 a 1.
Kód se čte od nově vzniklého uzlu.

Huffmanův kód je optimální.

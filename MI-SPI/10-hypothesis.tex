\section{Přednáška 10 -- Testování hypotéz}

\begin{description}
    \item[Hypotéza:] Vektor $X=(X_1, \dots)$, který má neznámé rozdělení -- tvrzení o rozdělení (jehož platnost neznáme) nazýváme hypotézou,
    \item[Testování hypotéz:] Mechanismus, jak na základě pozorovaných hodnot $X$ ověřit platnost hypotézy,
    \item[Nulová hypotéza:] $H_0$ označuje tvrzení, o kterém chceme rozhodnout,
    \item[ALternativní hypotéza:] $H_A$ opačné tvrzení proti $H_0$
\end{description}

Rozhodovacím procesem buď zamítneme nebo nezamítneme $H_0$.

\subsection{Chyby při zamítání hypotéz}

\subsubsection*{Chyba prvního druhu}

Zamítneme $H_0$, ačkoliv platí.

\subsubsection*{Chyba druhého druhu}

Nezamítneme $H_0$, ačkoli neplatí a platí $H_A$.

Nelze kontrolovat pravděpodobnosti obou těchto chyb, proto se $H_0$ volí tak, aby chyba \textit{prvního druhu} byla \textit{závažnější} než chyba druhého druhu.
Volíme $\alpha$ (hladina významnosti, běžně $1\%$ nebo $5\%$) a chceme, aby chyba prvního druhu byla nejvýše tato $alpha$.
Snažíme se konstruovat takové testy, aby chyba druhého druhu byla co nejmenší.

Zamítnutí $H_0$ je \textit{silný} výsledek.

\subsection{p-hodnota}

Minimální hladina významnosti, na které lze hypotézu $H_0$ zamítnout.
Je-li $p$-hodnota menší než požadovaná $\alpha$, zamítneme $H_0$

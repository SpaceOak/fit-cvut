\section{Přednáška 3 -- Náhodné vektory}

Na jednom výsledku experimentu lze měřit několik náhodných veličin najednou.
Uspořádanou n-tici nazveme \textit{náhodným vektorem}.
Obdobně dva rozměry $m$ a $n$ nazýváme vzniklou matici \textit{náhodnout maticí}.

\subsection{Náhodný vektor}

Pro $n \in \mathbb{N}$ uvažujeme náhodné veličiny $X_1, X_2, \dots, X_n$ na steném pravděpodobnostním oprostoru \pspace.
Vektor $X = (X_1, X_2, \dots, X_n)^T$ nazýváme náhodným vektorem.

Obdobnve lze zadefinovat náhodnout matici -- definována je pro $m,n \in \mathbb{Z}$ jako $\mathbf{Z} = (Z_{i,j})$.

\subsection{Náhodné vektory -- sdružené rozdělení}

Jednotlivé veličiny náhodného vektoru mohou mít různé rozdělení a hodnoty na sobě mohou nějakým záviset.

\subsubsection*{Sdružené rozdělení}

Pro náhodné veličiny z náhodného vektoru je tedy vhodné zkoumat jejich rozdělení společně jak \textit{sdružené rozdělení}.

Rozdělení se popisuje pomocí \textit{sdružené distribuční funkce} $F_X: \mathbb{R}^n \rightarrow \mathbb{R}$ danou vztahem:

$$
    F_X(x) \equiv F_X(x_1, \dots, x_n) = P(X_1 \leq x_1, \dots, X_n \leq x_n)
$$

pro každé $x \in \mathbb{R}^n$.

\subsubsection*{Sdružené diskrétní rozdělení}

Definiváno analogicky jako u náhodných veličin:

$$
    \sum_{x\in X}{(P_X = x)} \equiv \sum_{x \in X}{P(X_1 = x_1, \dots, X_n = x_n)} = 1
$$

Vizualizovat lze pomocí tabulky:

\begin{center}
\begin{tabular}{c|c c c}
    $P(X = x, Y = y)$ & $0.5$ & $1$ & $2$ \\ \hline
    2 & $0.3$ & $0.06$ & $0.04$ \\
    1 & $0.4$ & $0.15$ & $0.05$
\end{tabular}
\end{center}

\subsubsection*{Sdružené spojité rozdělení}

Pokud existuje nezáporná funkce $f_x: \mathbb{R}^n \rightarrow \mathbb{R}$:

$$
    F_X(x) = \int_{-\infty}^{x_1}{\dots} \int_{-\infty}^{x_n}{f_X(u)\textrm{d}u_1\dots\textrm{d}u_n}.
$$

kde $f_X$ je sdružená hustota pravděpodobnosti náhodného vektoru $X$.

\subsection{Marginální rozdělení}

Umožňuje ze známého sdruženého rozdělení vektoru $X = (X_1, \dots, X_n)$ zjistit rozdělení pouze části veličin $X$.

Rozdělení existuje pro každé $k \leq n$

\subsection{Nezávislost náhodných veličin}

Diskrétní veličiny$ X_1, \dots, X_n$ jsou nezávislé, pokud pro všechna $x \in \mathbb{R}^n$ platí:
$$
    P(X=x) = \prod_{i=1}^{n}{P(X_i = x_i)}
$$

Speciálně pro $X$ a $Y$ platí:

$$
    P(X = x, Y = y) = P(X=x)P(Y=y)
$$

\subsection{Střední hodnota funkce}

Střední hodnota funkce $h$ náhodného vektory $X$:
$$
    Eh(X) = \sum_{x\in X}{h(x)P(X=x)}
$$

Pro spojité:
$$
    Eh(x) = \int_{-\infty}^{\infty}{\dots}\int_{-\infty}^{\infty}{h(x)f_X(x)\textrm{d}x_1 \dots \textrm{d}x_n}
$$

\subsection{Kovariace}

Míra vzájemné lineární závislosti náhodných veličin $X$ a $Y$ lze zjistit pomocí korelačního koeficientu:

\begin{description}
    \item[Kovariace:] $\textrm{cov}(X,Y) = EXY - EXEY$
    \item[Korelační koeficient:] $\rho(X,Y) = \frac{cov(X,Y)}{\sqrt{var X}\sqrt{var Y}}$
\end{description}

Veličiny jsou nekorelované, pokud $cov(X,Y) = 0$ nebo také $EXY = EXEY$.
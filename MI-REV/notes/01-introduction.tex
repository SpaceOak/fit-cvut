\section{Přednáška 1 -- Úvod do RE}

Zabývá se procesem analýzy předmětného systému k vytvoření systému na vyšší úrovni abstrakce.
Zkoumá softwarový produkt jak funguje:

\begin{itemize}
    \item za účelem zjištění použitých algoritmů, metod, souborvých formátů a protokolů,
    \item za účelem nalezení zranitelnosti,
    \item návrh konkurenčního software,
    \item za účelem spolupráce (interakce) s programem,
    \item za účelem rozhodnutí o škodlivosti.
\end{itemize}

Při kompilaci (zdrojový kód \(\rightarrow\) objektový kód $\rightarrow$ spustitelný kód) dochází ke ztrátě imformací (komentáře, oddělené soubory, oddělené knihovny, debug informace, \dots).
Převést strojový kód zpět na zdrojový je nemožné, lze se ale přiblížit.

Program můžeme reonstruovat dvěma způsoby:

\begin{description}
    \item[Disassemblování] Převod do assembleru cílového procesoru. Velké množství kódu, zpravidla nutnost převodu do vyššího jazyka. Může být komplikováno díky obfuskacím.
    \item[Dekompilace] Převod do spustitelného kódu vyššího jazyka (C++, Ptyhon, \dots). Není zaručená správnost, proces komplikován packery, obfuskací. 
\end{description}

\subsection*{Analýza mrtvého kódu}

Provádí se na neběžícím spustitelném kódu s cílem prostudovat chování programu.
Prováděna pomocí disassembleru.
Snaha kód co nejvíce zmenšit před manuální analýzou (extrahování informací -- třídy, řetězce, RTTI, výjimky, volání std API, \dots).

\subsection*{Analýza živého kódu}

Prováděna na běžícím kódu s cílem zdokumentovat ho.
Typicky se provádí pomocí debuggeru.

Pomocí debuggeru je možné nastavit breakpointy, watchpointy a sledovat hodnoty registrů.
Program se ale může bránit pomocí zabijáckých vláken, detekce debuggeru nebo odmítnutí připojení.

\subsection*{ABI}

\textit{Application Binary Interface} (volací konvence) -- standard chování programu na cílové platformě aby byl kompatibilní s OS a binárním kódem jiných výrobců.
Jedná se o standard předávání argumentů, zarovnání dat na zásobníku, používání registrů, zpracování výjimek, identifikace typů pomocí RTTI, \dots

\subsection*{Name Mangling}

Dekorování funkcí/operátorů/metod v jazycích, které podporují jejich přetěžování.
Dekorace obsahuje jméno symbolu, jmenné prostory, \texttt{const}, 

\subsection*{Prolog}

Prvních několik instrukcí každé funkce, vytváří rámec, alokuje lokální proměnné, zajistí zarovnání zásobníku a zálohuje registry, které se nesmí změnit.
Při vstupu do funkce je na zásobník uložena návratová adresa (ukazuje na ni ESP registr).
Následuje prolog.

Struktura zásobníku je při adresaci:
\begin{description}
    \item[EBP-1..] Odkazuje na lokální proměnné,
    \item[EBP+0..EBP+3] Odkazuje na hodnotu EBP,
    \item[EBP+4..EBP+7] Odkazuje na návratovou hodnotu,
    \item[EBP+8..] Okazuje na argumenty funkce.    
\end{description}

\subsubsection*{Kanárek}

Náhodně generované slovo (pro program nebo každou funkci zlvášť), které je při návratu z funkce kontrolováno proti změnám.
Zamezuje se tak manipulaci se stackem pomocí nečekaného vstupu.
Kontroluje se pomocí instrukce XOR: \texttt{xor [ebp-4], ebp}.
Vkládá se pomocí \texttt{xor [security\_cookie], ebp; mov [ebp-4], [security\_cookie]}.

\subsection*{Epilog}

Poslední instrukce před příkazem \texttt{ret}.
Ruší rámec zásobníku a kontrolují \uv{kanárka}.

\subsection*{Strukturovaná obsluha výjimek}

Pokud dojde k výjimce na systémové úrovni (dělení nulou, dereference \texttt{NULL}) je zavolán handler.
Má strukturu:

\begin{itemize}
    \item registrace předchozí výjimky: \texttt{* prev},
    \item ukazatel na handler (funkci): \texttt{handler}.
\end{itemize}

Na první výjimku odkazuje registr FS (FS:[0]).
Pokud tento handler neumí výjimku obsloužit, přechází na další pomocí ukazatele \texttt{prev} dokud nějaký výjimku nezpracuje.
Pokud na žádný nenarazí použije OS svůj, kterým program ukončí nebo připojí debugger.

\img{seh-install.png}{Standardní instalace výjimek}{seh-install}{1}
\img{seh-uninstall.png}{Standardní odstranění výjimek}{seh-uninstall}{1}

Funkce zpracovávající výjimky bere argumenty:

\begin{description}
    \item[ExceptionRecord] informace o výjimce (např. stavový kód),
    \item[EstablisherFrame] odkazuje na registraci výjimky v rámci kde nastala,
    \item[ContextRecord] stav vlákna.
\end{description}

Po obsluze se řetězec prochází znovu a dochází k čištění zásobníku všech rámců, kde nebyla výjimka odchycena.

Útočník může přepsat handler instalované obsluhy a vyvolat výjimku.
Tím docílí spuštění svého kódu.

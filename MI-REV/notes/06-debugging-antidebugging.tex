\section{Debugging a antidebugging}

Debugger je nástroj pro hledání chyb za běhu aplikace -- dovoluje zkoumat na úrovni strojového kódu nebo disassembly.
Umožňuje spustit aplikaci, krokovat v ní, nastavit proměnné, vyhodnocovat jména symbolů, zpětný debug (návra zpět v kódu) a vzdálené ladění.
Rozlišujeme několik typů:

\begin{description}
    \item[V uživatelském režimu] ladění běžných aplikací,
    \item[Jádrový] umožňuje ladění jádra a ovladačů,
    \item[Zdrojového kódu] integrovaný do IDE, umožňuje debug na úrovni zdrojového kódu (řádky),
    \item[Nízkoúrovňový] na úrovni assembleru.
\end{description}

Debugger si buď vytvoří debugovaný proces (debugee) nebo se připojí k existujícímu.
Zpracovává události v ladící smyčce.

\subsection*{Připojení debuggeru}

\begin{description}
    \item[Neinvazivní] Nepřipojí se k aplikaci, jen přečte stav z pozastavených vláken -- pouze částečná kontrola,
    \item[Invazivní] aplikace má připojený právě jeden -- jsou mu posílány debugovací události (create process, create thread, load ddl, exit, \dots).
\end{description}

\subsection*{Trasování (Trace)}

Umožňuje sledovat tok řízení programu -- buď na úrovni funkcí nebo instrukcí.
Spomaluje vlákna, protože debugger musí po každé instrukci aktualizovat log.
Existují API na čtení a zápis do paměti procesu, stejně tak je možné pomocí kontextu zkoumat aktuální registry (a zapisovat do nich).

\subsection*{Softwarové Breakpointy}

Přerušení/výjimky vyvolané softwarem informujícím jádro OS o nutné změně toku programu.
Obsluha se bereze z registru \texttt{FS:[0]} -- hledání vhodného handleru.

Instalace probíhá pomocí přepsání prvního byte na zvolené adrese.
Při obsluze snížíme EIP o 1 a vrátíme původní byte.

Lze ho lehce detekovat, ale má neomezeně breakpointů, může měnit pamět a detekovat spuštění instrukce.

\subsection*{Hardwarové breakpointy}

Na architektuře x86 jich existuje 6.
Např. na sdělení aplikaci debugovací situace (DR6), debug control (DR7 - obsahuje příznay přeručsení), DR0-DR3 s adresou HW breakpointu.

Je podporován většinou debuggerů, nelze ho tak jednoduše detekovat (nemění obsah paměti).
Může použít omezené množství breakpointů.


\subsection*{Debugování jádra}

Není jednoduché jako lokální -- potřeba řídit z jiného stroje, aby bylo možné jádro pozastavit.
Veškerá paměť se stále mění -- kernel stále běží, nejsou dostupné breakpointy ani práce se zásobníkem/registry.

\subsection*{Antidebugging}

Obrana proti debugovaní.
Aplikace detekuje připojení debuggeru a snaží se mu uniknout (nebo ho ukončit).

Lze použít standardní API pro detekci přítomnosti debuggeru -- debugger ale může i tuto hodnotu ovlivnit.

Při připojení je navíc v interní struktuře \texttt{ProcesHeap}, pole \texttt{0x10} od začátku je \texttt{ForceFlags} -- při vytvoření haldy debuggerem je nenulové -- lze spustit s vypnutou haldou.

Globální příznak \texttt{NTGlobalFlag} -- má nastavené bity ale nelze se na to spolehnout.
Bránit se dá přepsáním.

Proces může sledovat další spuštěné procesy -- pokud je spuštěný debugger může na to reagovat.

Detekce se dá dělat pomocí časování -- měří se čas volání funkcí -- při spuštěném debuggeru je delší.

Je možné navíc donutit debugujícího uživatele vypnout upozornění na výjimky a následně v jejich obsluze schovat důležitý kód.

\subsection*{Únik z debuggeru}

\begin{itemize}
    \item proces se připojí sám k sobě jako debugger,
    \item injektáž kódu do jiného procesu,
    \item schování kódu do handleru výjimky (do řetězce pak lze vložit stovky falešných handlerů),
    \item spuštění kódu v \texttt{initterm} před vlastní metodou main.
\end{itemize}

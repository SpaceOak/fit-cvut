\section{Přednáška 8 -- Architektonické vzory, sdílení paměti a zasílání zpráv}

\subsection{Sdílení paměti v rámci aplikace}

\subsubsection{Database centric}

Architektura, v níž databáze hrají zásadní roli.
Typ klient-server architektury, kde server obsluhuje veškeré CRUD operace.
Logiku aplikace je možné následně implementovat jako:

\begin{description}
    \item[table-driven] Chování aplikace závisí na obsahu databáze
    \item[stored-procedures] Aplikace volají procedury, které jsou na databázovém serveru (např. Oracle PL SQL)
    \item[communicating] Paralelní procesy využívají databázi pro komunikaci mezi sebou.
\end{description}

\subsubsection{Rule based}

\textit{Interpretuje pravidla ve formě \texttt{IF condition THEN action}, kde podmínka testuje pracovní paměť (např. na existenci symbolů, dat, \dots).}

\subsubsection{Blackboard}

Návrh systému, který integruje rozsáhlé moduly a komplexní nedeterministickou strategii ovládání.

\subsection{Messaging}

\subsubsection{Event-driven}

Architektura propagující změny stavu pomocí eventů.

\begin{description}
    \item[Event] Význačná změna stavu
    \item[Event notification] Propagování informací o změně
\end{description}

Pomocí mediatoru je navštíven každý objekt, který se zaregistruje o informace o změně stavu.

\subsubsection{Publish-subscribe}

Architektura, kde \texttt{Publisher} zasílá zprávy specifické kategorie na server, kde se o zprávy těchto kategorií mohou zaregistrovat \texttt{Subscriber} objekty.

\begin{description}
    \item[Topic-based] \texttt{Publisher} zasílá zprávy do pojmenovaných kanálů, o jejichž existenci se musí postarat. Každý \texttt{Subscriber} dostane stejnou zprávu.
    \item[Content-based] \texttt{Subscriber} si definuje omezení na zprávy, které chce přijímat.
\end{description}

\subsubsection{Asynchronní zprávy}

Vrstva, která umožňuje heterogenním komponentám asynchronně i přes jejich (implementační) odlišnosti.
Implementováno pomocí front (např. JMS).
